%----------------------------------------------------------------
%
%  File    :  intro.tex
%
%  Authors :  Wolfgang Radinger-Peer
% 
%  Created :  3. Feb 2019
% 
%  Changed :  3. Feb 2019
% 
%----------------------------------------------------------------

\chapter{\textsc{Introduction}}
\label{chap:intro}

\section{Context and previous research }
\label{sec:context}


An intro that will lead into the theoretical framework which follows in the next chapter.

\section{Headings}
\label{sec:heading}

Headings structure a text and demonstrate the importance of individual sections. Different levels of structure must be indicated by different heading levels. 
Paragraphs that are of equal importance receive headings of the same level.

\subsection{Creating headings}
\label{sec:createheadings}

The headings in this chapter as well as the other chapters will largely be dependent on your research. The introductory chapter serves as a presentation and an overview of your thesis. You need to make the reader familiar with the topics that you will investigate, and prepare him/her for what is forthcoming throughout the next chapters. 

\section{Formatting throughout the thesis}
This section will describe the formatting that should be applied throughout the thesis.

\subsection{Fonts and alignment}
\label{sec:fonts}
The font used should be consistent throughout the thesis. The default font for \LaTeX\ is Calibri (11pt) which is accepted, 
as well as other sans fonts such as Arial and Helvetica (11pt). 
Paragraphs should always be separated with a line break (but not with a blank line), and the text should be justified. 

\subsection{Margins and spacing}
\label{sec:margins}

The margins preferred for the thesis are defaults in \LaTeX\:
\begin{itemize}
\item Left and right margins: 1 inch (2.5 cm)
\item Top and bottom margins: 1 inch (2.5 cm)
\item Back margin: 0.4 inches (1 cm)
\end{itemize}
	
The thesis should have line spacing of: “Multiple 1.3” and a spacing before each paragraph of 12 pt. 

\subsection{Page numbering and printing}
\label{sec:pagenumbering}

Pages should be numbered throughout the thesis. The page number is on even pages left aligned, on uneven pages right aligned. The thesis should be printed on both sides to save paper. 

\subsection{Referencing in text/citation}
\label{sec:referencing}
The thesis should be fully referenced, and any material used without providing a reference is considered as plagiarism. 
To see the university guidelines on plagiarism, please go to \url{http://www.modul.ac.at/sites/modul/files/Plagiarism.pdf}.

Direct, literal citation is always reproduced to the letter and set in “quotation marks.” As a general rule, longer citations are – if they are absolutely necessary – 
formatted differently for a better accentuation and displayed with an indent. Indirect, analogous reference to citations (“paraphrases”) must also be 
labeled (mainly with “cf.”), since it involves an external body of thought despite there being a description in your own words. 

The following list serves as a 
guideline about how to quote different sources using APA formatting. 
This system does not use footnotes, but it offers a brief note in the text about where the information comes from.

\begin{description}
	\item[The work of an individual author:]\hfill \\
	... previous pre-occupation with this phenomenon (Müller, 1954)...\hfill \\
	... Müller (1954) already dealt with this phenomenon ...
%
	\item[A particular page or pages should be specified:]\hfill \\
	... Kristofferson (1990, p. 268) offers an alternative explanation...\hfill \\
	... An alternative explanation is offered by Kristofferson (1990, pp. 268-70)....
%
	\item[The work of two authors:]\hfill \\
	...further analysis (Schmid \& Maier, 1973)...\hfill \\
	...according to Schmid \& Maier (1973),..
%
	\item[The work of several authors:]\hfill \\
	Only the name of the first author is stated, followed by “et al.“ and the year\hfill \\
	... Maier et al. (1981)
%	
	\item[Several works of different authors:]\hfill \\
	...This interpretation is disputed by several scholars (O’Keefe 1988; Joye 1989).
%
	\item[Two different works of the same author:]\hfill \\
	... Parton (1991a; 1991b) has carried out a number of research projects which reinforce these findings. \hfill \\
	... Parton (1991; 1996) has carried out a number of research projects which reinforce these findings.
%
	\item[Secondary Citations:] In the case of secondary citations, the original source is no longer available or accessible. The citation belonging to the author of the original research paper that is no longer avail-able appears in the text followed by parentheses or if this citation is already in parentheses, after a comma followed by the note “cited by” and details of the respective source that is available to the author. However, the original work that is not present must be cited in the bibliography.\hfill \\
... Müller (1954, cited by Barnabas, 1960) \hfill \\
... (Müller, 1954, cited by Barnabas, 1960)
%
	\item[Internet references:] The main rule to remember is that the citation should be matched to the reference in the bibliography. Thus, the citation should be made using the name/organization which is displayed in the bibliography.\hfill \\ ... (World Tourism Organisation, 2009). 
\end{description}

\section{Research aims and objectives (and hypotheses if applicable)}
\label{sec:aims}
Aims and objectives should be clearly stated. 

\section{Structure of thesis}
\label{sec:structure}
Under the last heading of the first chapter you should give a short description of the chapter layout of the thesis.

