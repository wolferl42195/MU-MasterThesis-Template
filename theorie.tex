%----------------------------------------------------------------
%
%  File    :  theorie.tex
%
%  Author :  Wolfgang Radinger-Peer
% 
%  Created :  3. Feb 2019
% 
%  Changed :  3. Feb 2019
% 
%----------------------------------------------------------------

\chapter{\textsc{Literature Review}}
\label{chap:theory}

\section{Introduction}
\label{sec:introcap2}

Overview of chapter and linking with previous chapter

\section{Theoretical framework (if applicable)}
\label{sec:introcap2}

A diagram showing how the different topics within the literature relate, and how they come together in your research

\section{First relevant topic within literature}
\label{sec:rel1}

\subsection{Sub-topic}
\label{sec:rel1sub1}

\subsection{Sub-topic}
\label{sec:rel1sub2}

\cite{Fowler72s}


\section{Second relevant topic... etc}
\label{sec:rel2}

Etc… the number of relevant topics and sub-topics will vary depending on the nature and scope of your research. What is important to remember is that all 
relevant topics should be included in the literature review, and serve as an introduction to your research. Later in the thesis you will refer your findings back to
 relevant literature in this chapter, so it needs to cover all the literature that is of importance to your findings. 
In addition, you should state how your thesis will contribute to the current knowledge, either by making this explicit under a separate heading or by 
mentioning the contribution of the thesis throughout the chapter where relevant additions will be made. 

\section{Conclusion}

Short summary and link to next chapter

